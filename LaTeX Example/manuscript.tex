\documentclass[11pt]{article}

\usepackage{fullpage}
\usepackage{url}
\usepackage{xcolor}
\definecolor{darkgreen}{RGB}{0,100,0}
\usepackage[hidelinks]{hyperref}

\hypersetup{
  colorlinks=true,
  linkcolor=darkgreen,
  citecolor=darkgreen,
  urlcolor=blue
}

\title{Sorting Network \LaTeX\ Test}

\author{Ian Parberry\\
\url{https://ianparberry.com}}
\date{}

\begin {document}
  \maketitle
	
	\noindent
	Sorting networks exported in \TeX\ format
	from the \verb'WinVerifyAndDraw' tool
	in Parberry~\cite{Parberry2022SNS}
	contain a \verb'picture' environment
	that can be directly input to your \LaTeX\ file using the \verb"\input"
	command.
  For example, suppose you have exported a sorting network
	to file \verb"exported.tex". Use the following to incorporate this file
	as a figure in your \LaTeX\ document. 
	
\begin{verbatim}
  \begin{figure}
    \centering\input{exported.tex}
  \end{figure}
\end{verbatim}
	
	\noindent
	The remainder of this document will show you how it looks using
	the exported files
	\verb"knuth16h.tex" (exported in \verb'Horizontal' draw mode)
	in Figure~\ref{fig-h} and
	\verb"knuth16v.tex" (exported in \verb'Vertical' draw mode) in Figure~\ref{fig-v}.
	
	Knuth~\cite{knuth1997art} depicts sorting networks with the channels
	drawn horizontally and the comparators drawn vertically (see,
	for example, Figure~\ref{fig-h}). Each comparator
	directs the minimum of its two inputs to the upper channel and the 
	maximum to the lower channel. The inputs are presented at the left
	and the outputs appear at the right in non-decreasing order from top to bottom.

	\begin{figure}[h!]
		\centering\setlength{\unitlength}{0.5pt}
\begin{picture}(576,370)(0,-370)
\thicklines
\put(0,-4){\line(1,0){576}}
\put(0,-28){\line(1,0){576}}
\put(0,-52){\line(1,0){576}}
\put(0,-76){\line(1,0){576}}
\put(0,-100){\line(1,0){576}}
\put(0,-124){\line(1,0){576}}
\put(0,-148){\line(1,0){576}}
\put(0,-172){\line(1,0){576}}
\put(0,-196){\line(1,0){576}}
\put(0,-220){\line(1,0){576}}
\put(0,-244){\line(1,0){576}}
\put(0,-268){\line(1,0){576}}
\put(0,-292){\line(1,0){576}}
\put(0,-316){\line(1,0){576}}
\put(0,-340){\line(1,0){576}}
\put(0,-364){\line(1,0){576}}
\put(24,-28){\circle*{8}}
\put(24,-4){\circle*{8}}
\put(24,-4){\line(0,-1){24}}
\put(24,-76){\circle*{8}}
\put(24,-52){\circle*{8}}
\put(24,-52){\line(0,-1){24}}
\put(24,-124){\circle*{8}}
\put(24,-100){\circle*{8}}
\put(24,-100){\line(0,-1){24}}
\put(24,-172){\circle*{8}}
\put(24,-148){\circle*{8}}
\put(24,-148){\line(0,-1){24}}
\put(24,-220){\circle*{8}}
\put(24,-196){\circle*{8}}
\put(24,-196){\line(0,-1){24}}
\put(24,-268){\circle*{8}}
\put(24,-244){\circle*{8}}
\put(24,-244){\line(0,-1){24}}
\put(24,-316){\circle*{8}}
\put(24,-292){\circle*{8}}
\put(24,-292){\line(0,-1){24}}
\put(24,-364){\circle*{8}}
\put(24,-340){\circle*{8}}
\put(24,-340){\line(0,-1){24}}
\put(48,-52){\circle*{8}}
\put(48,-4){\circle*{8}}
\put(48,-4){\line(0,-1){48}}
\put(48,-148){\circle*{8}}
\put(48,-100){\circle*{8}}
\put(48,-100){\line(0,-1){48}}
\put(48,-244){\circle*{8}}
\put(48,-196){\circle*{8}}
\put(48,-196){\line(0,-1){48}}
\put(48,-340){\circle*{8}}
\put(48,-292){\circle*{8}}
\put(48,-292){\line(0,-1){48}}
\put(64,-76){\circle*{8}}
\put(64,-28){\circle*{8}}
\put(64,-28){\line(0,-1){48}}
\put(64,-172){\circle*{8}}
\put(64,-124){\circle*{8}}
\put(64,-124){\line(0,-1){48}}
\put(64,-268){\circle*{8}}
\put(64,-220){\circle*{8}}
\put(64,-220){\line(0,-1){48}}
\put(64,-364){\circle*{8}}
\put(64,-316){\circle*{8}}
\put(64,-316){\line(0,-1){48}}
\put(88,-100){\circle*{8}}
\put(88,-4){\circle*{8}}
\put(88,-4){\line(0,-1){96}}
\put(88,-292){\circle*{8}}
\put(88,-196){\circle*{8}}
\put(88,-196){\line(0,-1){96}}
\put(104,-124){\circle*{8}}
\put(104,-28){\circle*{8}}
\put(104,-28){\line(0,-1){96}}
\put(104,-316){\circle*{8}}
\put(104,-220){\circle*{8}}
\put(104,-220){\line(0,-1){96}}
\put(120,-148){\circle*{8}}
\put(120,-52){\circle*{8}}
\put(120,-52){\line(0,-1){96}}
\put(120,-340){\circle*{8}}
\put(120,-244){\circle*{8}}
\put(120,-244){\line(0,-1){96}}
\put(136,-172){\circle*{8}}
\put(136,-76){\circle*{8}}
\put(136,-76){\line(0,-1){96}}
\put(136,-364){\circle*{8}}
\put(136,-268){\circle*{8}}
\put(136,-268){\line(0,-1){96}}
\put(160,-196){\circle*{8}}
\put(160,-4){\circle*{8}}
\put(160,-4){\line(0,-1){192}}
\put(176,-220){\circle*{8}}
\put(176,-28){\circle*{8}}
\put(176,-28){\line(0,-1){192}}
\put(192,-244){\circle*{8}}
\put(192,-52){\circle*{8}}
\put(192,-52){\line(0,-1){192}}
\put(208,-268){\circle*{8}}
\put(208,-76){\circle*{8}}
\put(208,-76){\line(0,-1){192}}
\put(224,-292){\circle*{8}}
\put(224,-100){\circle*{8}}
\put(224,-100){\line(0,-1){192}}
\put(240,-316){\circle*{8}}
\put(240,-124){\circle*{8}}
\put(240,-124){\line(0,-1){192}}
\put(256,-340){\circle*{8}}
\put(256,-148){\circle*{8}}
\put(256,-148){\line(0,-1){192}}
\put(272,-364){\circle*{8}}
\put(272,-172){\circle*{8}}
\put(272,-172){\line(0,-1){192}}
\put(296,-52){\circle*{8}}
\put(296,-28){\circle*{8}}
\put(296,-28){\line(0,-1){24}}
\put(296,-292){\circle*{8}}
\put(296,-76){\circle*{8}}
\put(296,-76){\line(0,-1){216}}
\put(296,-340){\circle*{8}}
\put(296,-316){\circle*{8}}
\put(296,-316){\line(0,-1){24}}
\put(312,-196){\circle*{8}}
\put(312,-100){\circle*{8}}
\put(312,-100){\line(0,-1){96}}
\put(328,-244){\circle*{8}}
\put(328,-124){\circle*{8}}
\put(328,-124){\line(0,-1){120}}
\put(344,-220){\circle*{8}}
\put(344,-148){\circle*{8}}
\put(344,-148){\line(0,-1){72}}
\put(360,-268){\circle*{8}}
\put(360,-172){\circle*{8}}
\put(360,-172){\line(0,-1){96}}
\put(384,-100){\circle*{8}}
\put(384,-28){\circle*{8}}
\put(384,-28){\line(0,-1){72}}
\put(384,-292){\circle*{8}}
\put(384,-124){\circle*{8}}
\put(384,-124){\line(0,-1){168}}
\put(400,-196){\circle*{8}}
\put(400,-52){\circle*{8}}
\put(400,-52){\line(0,-1){144}}
\put(400,-340){\circle*{8}}
\put(400,-268){\circle*{8}}
\put(400,-268){\line(0,-1){72}}
\put(416,-220){\circle*{8}}
\put(416,-76){\circle*{8}}
\put(416,-76){\line(0,-1){144}}
\put(432,-244){\circle*{8}}
\put(432,-148){\circle*{8}}
\put(432,-148){\line(0,-1){96}}
\put(448,-316){\circle*{8}}
\put(448,-172){\circle*{8}}
\put(448,-172){\line(0,-1){144}}
\put(472,-100){\circle*{8}}
\put(472,-52){\circle*{8}}
\put(472,-52){\line(0,-1){48}}
\put(472,-196){\circle*{8}}
\put(472,-148){\circle*{8}}
\put(472,-148){\line(0,-1){48}}
\put(472,-292){\circle*{8}}
\put(472,-244){\circle*{8}}
\put(472,-244){\line(0,-1){48}}
\put(488,-124){\circle*{8}}
\put(488,-76){\circle*{8}}
\put(488,-76){\line(0,-1){48}}
\put(488,-220){\circle*{8}}
\put(488,-172){\circle*{8}}
\put(488,-172){\line(0,-1){48}}
\put(488,-316){\circle*{8}}
\put(488,-268){\circle*{8}}
\put(488,-268){\line(0,-1){48}}
\put(512,-148){\circle*{8}}
\put(512,-76){\circle*{8}}
\put(512,-76){\line(0,-1){72}}
\put(512,-244){\circle*{8}}
\put(512,-172){\circle*{8}}
\put(512,-172){\line(0,-1){72}}
\put(528,-196){\circle*{8}}
\put(528,-124){\circle*{8}}
\put(528,-124){\line(0,-1){72}}
\put(528,-292){\circle*{8}}
\put(528,-220){\circle*{8}}
\put(528,-220){\line(0,-1){72}}
\put(552,-100){\circle*{8}}
\put(552,-76){\circle*{8}}
\put(552,-76){\line(0,-1){24}}
\put(552,-148){\circle*{8}}
\put(552,-124){\circle*{8}}
\put(552,-124){\line(0,-1){24}}
\put(552,-196){\circle*{8}}
\put(552,-172){\circle*{8}}
\put(552,-172){\line(0,-1){24}}
\put(552,-244){\circle*{8}}
\put(552,-220){\circle*{8}}
\put(552,-220){\line(0,-1){24}}
\put(552,-292){\circle*{8}}
\put(552,-268){\circle*{8}}
\put(552,-268){\line(0,-1){24}}
\end{picture}

		\caption{A 16-input sorting network from Knuth~\cite{knuth1997art}.}
		\label{fig-h}
	\end{figure}
	
	Parberry~\cite{parberry1990single, parberry1991computational, parberry1992pairwise}
	depicts sorting networks with the channels
	drawn vertically and the comparators drawn horizontally (see,
	for example, Figure~\ref{fig-v}). Each comparator
	directs the minimum of its two inputs to the left channel and the 
	maximum to the right channel. The inputs are presented at the top
	and the outputs appear at the bottom in non-decreasing order from left to right.
	
  \begin{figure}[h!]
	  \centering\setlength{\unitlength}{0.5pt}
\begin{picture}(370,576)(0,-576)
\thicklines
\put(4,-0){\line(0,-1){576}}
\put(28,-0){\line(0,-1){576}}
\put(52,-0){\line(0,-1){576}}
\put(76,-0){\line(0,-1){576}}
\put(100,-0){\line(0,-1){576}}
\put(124,-0){\line(0,-1){576}}
\put(148,-0){\line(0,-1){576}}
\put(172,-0){\line(0,-1){576}}
\put(196,-0){\line(0,-1){576}}
\put(220,-0){\line(0,-1){576}}
\put(244,-0){\line(0,-1){576}}
\put(268,-0){\line(0,-1){576}}
\put(292,-0){\line(0,-1){576}}
\put(316,-0){\line(0,-1){576}}
\put(340,-0){\line(0,-1){576}}
\put(364,-0){\line(0,-1){576}}
\put(28,-24){\circle*{8}}
\put(4,-24){\circle*{8}}
\put(4,-24){\line(1,0){24}}
\put(76,-24){\circle*{8}}
\put(52,-24){\circle*{8}}
\put(52,-24){\line(1,0){24}}
\put(124,-24){\circle*{8}}
\put(100,-24){\circle*{8}}
\put(100,-24){\line(1,0){24}}
\put(172,-24){\circle*{8}}
\put(148,-24){\circle*{8}}
\put(148,-24){\line(1,0){24}}
\put(220,-24){\circle*{8}}
\put(196,-24){\circle*{8}}
\put(196,-24){\line(1,0){24}}
\put(268,-24){\circle*{8}}
\put(244,-24){\circle*{8}}
\put(244,-24){\line(1,0){24}}
\put(316,-24){\circle*{8}}
\put(292,-24){\circle*{8}}
\put(292,-24){\line(1,0){24}}
\put(364,-24){\circle*{8}}
\put(340,-24){\circle*{8}}
\put(340,-24){\line(1,0){24}}
\put(52,-48){\circle*{8}}
\put(4,-48){\circle*{8}}
\put(4,-48){\line(1,0){48}}
\put(148,-48){\circle*{8}}
\put(100,-48){\circle*{8}}
\put(100,-48){\line(1,0){48}}
\put(244,-48){\circle*{8}}
\put(196,-48){\circle*{8}}
\put(196,-48){\line(1,0){48}}
\put(340,-48){\circle*{8}}
\put(292,-48){\circle*{8}}
\put(292,-48){\line(1,0){48}}
\put(76,-64){\circle*{8}}
\put(28,-64){\circle*{8}}
\put(28,-64){\line(1,0){48}}
\put(172,-64){\circle*{8}}
\put(124,-64){\circle*{8}}
\put(124,-64){\line(1,0){48}}
\put(268,-64){\circle*{8}}
\put(220,-64){\circle*{8}}
\put(220,-64){\line(1,0){48}}
\put(364,-64){\circle*{8}}
\put(316,-64){\circle*{8}}
\put(316,-64){\line(1,0){48}}
\put(100,-88){\circle*{8}}
\put(4,-88){\circle*{8}}
\put(4,-88){\line(1,0){96}}
\put(292,-88){\circle*{8}}
\put(196,-88){\circle*{8}}
\put(196,-88){\line(1,0){96}}
\put(124,-104){\circle*{8}}
\put(28,-104){\circle*{8}}
\put(28,-104){\line(1,0){96}}
\put(316,-104){\circle*{8}}
\put(220,-104){\circle*{8}}
\put(220,-104){\line(1,0){96}}
\put(148,-120){\circle*{8}}
\put(52,-120){\circle*{8}}
\put(52,-120){\line(1,0){96}}
\put(340,-120){\circle*{8}}
\put(244,-120){\circle*{8}}
\put(244,-120){\line(1,0){96}}
\put(172,-136){\circle*{8}}
\put(76,-136){\circle*{8}}
\put(76,-136){\line(1,0){96}}
\put(364,-136){\circle*{8}}
\put(268,-136){\circle*{8}}
\put(268,-136){\line(1,0){96}}
\put(196,-160){\circle*{8}}
\put(4,-160){\circle*{8}}
\put(4,-160){\line(1,0){192}}
\put(220,-176){\circle*{8}}
\put(28,-176){\circle*{8}}
\put(28,-176){\line(1,0){192}}
\put(244,-192){\circle*{8}}
\put(52,-192){\circle*{8}}
\put(52,-192){\line(1,0){192}}
\put(268,-208){\circle*{8}}
\put(76,-208){\circle*{8}}
\put(76,-208){\line(1,0){192}}
\put(292,-224){\circle*{8}}
\put(100,-224){\circle*{8}}
\put(100,-224){\line(1,0){192}}
\put(316,-240){\circle*{8}}
\put(124,-240){\circle*{8}}
\put(124,-240){\line(1,0){192}}
\put(340,-256){\circle*{8}}
\put(148,-256){\circle*{8}}
\put(148,-256){\line(1,0){192}}
\put(364,-272){\circle*{8}}
\put(172,-272){\circle*{8}}
\put(172,-272){\line(1,0){192}}
\put(52,-296){\circle*{8}}
\put(28,-296){\circle*{8}}
\put(28,-296){\line(1,0){24}}
\put(292,-296){\circle*{8}}
\put(76,-296){\circle*{8}}
\put(76,-296){\line(1,0){216}}
\put(340,-296){\circle*{8}}
\put(316,-296){\circle*{8}}
\put(316,-296){\line(1,0){24}}
\put(196,-312){\circle*{8}}
\put(100,-312){\circle*{8}}
\put(100,-312){\line(1,0){96}}
\put(244,-328){\circle*{8}}
\put(124,-328){\circle*{8}}
\put(124,-328){\line(1,0){120}}
\put(220,-344){\circle*{8}}
\put(148,-344){\circle*{8}}
\put(148,-344){\line(1,0){72}}
\put(268,-360){\circle*{8}}
\put(172,-360){\circle*{8}}
\put(172,-360){\line(1,0){96}}
\put(100,-384){\circle*{8}}
\put(28,-384){\circle*{8}}
\put(28,-384){\line(1,0){72}}
\put(292,-384){\circle*{8}}
\put(124,-384){\circle*{8}}
\put(124,-384){\line(1,0){168}}
\put(196,-400){\circle*{8}}
\put(52,-400){\circle*{8}}
\put(52,-400){\line(1,0){144}}
\put(340,-400){\circle*{8}}
\put(268,-400){\circle*{8}}
\put(268,-400){\line(1,0){72}}
\put(220,-416){\circle*{8}}
\put(76,-416){\circle*{8}}
\put(76,-416){\line(1,0){144}}
\put(244,-432){\circle*{8}}
\put(148,-432){\circle*{8}}
\put(148,-432){\line(1,0){96}}
\put(316,-448){\circle*{8}}
\put(172,-448){\circle*{8}}
\put(172,-448){\line(1,0){144}}
\put(100,-472){\circle*{8}}
\put(52,-472){\circle*{8}}
\put(52,-472){\line(1,0){48}}
\put(196,-472){\circle*{8}}
\put(148,-472){\circle*{8}}
\put(148,-472){\line(1,0){48}}
\put(292,-472){\circle*{8}}
\put(244,-472){\circle*{8}}
\put(244,-472){\line(1,0){48}}
\put(124,-488){\circle*{8}}
\put(76,-488){\circle*{8}}
\put(76,-488){\line(1,0){48}}
\put(220,-488){\circle*{8}}
\put(172,-488){\circle*{8}}
\put(172,-488){\line(1,0){48}}
\put(316,-488){\circle*{8}}
\put(268,-488){\circle*{8}}
\put(268,-488){\line(1,0){48}}
\put(148,-512){\circle*{8}}
\put(76,-512){\circle*{8}}
\put(76,-512){\line(1,0){72}}
\put(244,-512){\circle*{8}}
\put(172,-512){\circle*{8}}
\put(172,-512){\line(1,0){72}}
\put(196,-528){\circle*{8}}
\put(124,-528){\circle*{8}}
\put(124,-528){\line(1,0){72}}
\put(292,-528){\circle*{8}}
\put(220,-528){\circle*{8}}
\put(220,-528){\line(1,0){72}}
\put(100,-552){\circle*{8}}
\put(76,-552){\circle*{8}}
\put(76,-552){\line(1,0){24}}
\put(148,-552){\circle*{8}}
\put(124,-552){\circle*{8}}
\put(124,-552){\line(1,0){24}}
\put(196,-552){\circle*{8}}
\put(172,-552){\circle*{8}}
\put(172,-552){\line(1,0){24}}
\put(244,-552){\circle*{8}}
\put(220,-552){\circle*{8}}
\put(220,-552){\line(1,0){24}}
\put(292,-552){\circle*{8}}
\put(268,-552){\circle*{8}}
\put(268,-552){\line(1,0){24}}
\end{picture}

		\caption{A 16-input sorting network from Knuth~\cite{knuth1997art}
		  drawn in the style 
		  of Parberry~\cite{parberry1990single, parberry1991computational, parberry1992pairwise}.}
		\label{fig-v}
  \end{figure}
	
\bibliography{refs}
\bibliographystyle{abbrv}

\end{document}
